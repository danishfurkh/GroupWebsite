\documentclass[11pt]{article}
\usepackage{amsmath}
\usepackage{amsfonts}
\usepackage{rotating}
\usepackage[bf]{caption2} \captionstyle{normal}
\usepackage{url}
\usepackage{epsfig}
\usepackage[bookmarks=false, colorlinks=true, linkcolor=blue,
            citecolor=blue]{hyperref}
\usepackage{color}


\renewcommand\baselinestretch{1.15}
\setlength{\textwidth}{15.5cm} \setlength{\oddsidemargin}{0.5cm}
\setlength{\evensidemargin}{0.50cm} \setlength{\topmargin}{-1.0cm}
\setlength{\textheight}{23.cm} \setlength{\parindent}{0.0cm}

\def\sfc#1{\textcolor{red}{ \textbf{... #1 ...} } }          % comment
\def\sfq#1{\textcolor{red}{ \textbf{(?? #1 ??)} }}           % question
\def\sfi#1{\textcolor{red}{ $\bullet\;$ \textit{#1}}}        % insertion

\def\ket#1{ $    \left\vert   #1 \right\rangle $ }
\def\ketm#1{     \left\vert   #1 \right\rangle   }
\def\bra#1{ $    \left\langle #1 \right\vert   $ }
\def\bram#1{     \left\langle #1 \right\vert     }
\def\spr#1#2{ $  \left\langle #1 \left\vert \right. #2 \right\rangle $ }
\def\sprm#1#2{   \left\langle #1 \left\vert \right. #2 \right\rangle   }
\def\me#1#2#3{ $ \left\langle #1 \left\vert #2 \right\vert #3 \right\rangle $ }
\def\mem#1#2#3{  \left\langle #1 \left\vert #2 \right\vert #3 \right\rangle   }
\def\redme#1#2#3{ $ \left\langle #1 \left\Vert
                  #2 \right\Vert #3 \right\rangle $ }
\def\redmem#1#2#3{  \left\langle #1 \left\Vert
                  #2 \right\Vert #3 \right\rangle   }
%
\def\spinup{   $ \left( \begin{array}{c} 1 \\ 0 \end{array} \right) $ }
\def\spindown{ $ \left( \begin{array}{c} 0 \\ 1 \end{array} \right) $ }
\def\Itwo{   $ \left( \begin{array}{c c} 1 & 0 \\ 0 & 1 \end{array} \right) $ }
\def\Itwom{    \left( \begin{array}{c c} 1 & 0 \\ 0 & 1 \end{array} \right)   }
\def\sigmax{ $ \left( \begin{array}{c c} 0 & 1 \\ 1 & 0 \end{array} \right) $ }
\def\sigmaxm{  \left( \begin{array}{c c} 0 & 1 \\ 1 & 0 \end{array} \right)   }
\def\sigmaz{ $ \left( \begin{array}{c c} 1 & 0 \\ 0 &-1 \end{array} \right) $ }
\def\sigmazm{  \left( \begin{array}{c c} 1 & 0 \\ 0 &-1 \end{array} \right)   }
%
\def\proc#1{  {\bf #1} }
\def\subproc#1{  { #1} }
\def\procref#1{{\sf #1}}
\def\procvar#1{\texttt{ #1}}
\def\procargop{ {\small\bf Argument options:} }
\def\procadd{ {\small\bf Additional information:} }
\def\procout{ {\small\bf Output:} }
\def\procsee{ {\small\bf See also:} }
\def\procprop#1{ {\small\bf #1} }
\def\procskip{\medskip}
\def\procsection#1{\vspace*{0.1cm} 
                   \textcolor{red}{\small\underline{\textbf{#1}}} }
\def\procsectionref#1{\textcolor{red}{\small\textbf{#1}} }
\def\proccommand#1{\textcolor{blue}{#1}}
\def\map{{\small\bf *}\,}
\def\mleaf{ {\small\bf $ \spadesuit $ } }

\def\half{    \frac{1}{2}  }
\def\xhalf#1{ \frac{#1}{2} }
%
\def\etal{\textit{et al.}}
\def\elong{\sc }
%


\begin{document}


\title{\textbf{``Quantenprozesse und Quantenprotokolle''} \\[0.1cm]
       Ihr Thema hier bitte.}

\author{Name          \\
        \\
	Physikalisch-Astronomische Fakult\"a{}t \\[0.1cm]
        Universit\"a{}t Jena \\[0.1cm]
        \\
        }


\maketitle

\begin{abstract}
Hier bitte einen kurzen Abstrakt, evtl. auch auf Englisch.
\end{abstract}

\bigskip
\bigskip


%
%
%
%
%
\section{Einf\"u{}hrung/Introduction}

...


%
%
%
%
%
\begin{figure}[h]
\begin{center}
  \epsfig{file=b16.01-workout-fig.eps, height=5.5cm, angle=0}
  \caption{Elektromagnetisches Spektrum (aus Wikipedia).}
  \label{em-spectrum}
\end{center}
\end{figure}

%
%
%
%
%
\section{Theoretische Grundlagen/Theoretical background}

...


%
%
%
%
\subsection{Qubit-Darstellung}

%
\begin{equation}
   \ketm{\psi} \,=\,  a \ketm{0} + b \ketm{1} \,\equiv\,
   a \,       \left( {\begin{array}{*{20}c} {1 } \\ {0 } \end{array} } \right)
   \,+\, b \, \left( {\begin{array}{*{20}c} {0 } \\ {1 } \end{array} } \right)
   \,=\,
   \left( {\begin{array}{*{20}c}
   {a }  \\
   {b }  \\
   \end{array} } \right) \, ,
\end{equation}
%
in which the state $\ketm{\psi}$ ...


%
%
%
%
%
%
\begin{table}
\begin{small}
\caption{\rm Vergleich verschiedener Prozesse.}
\label{vortex-beams}
\begin{center}
\begin{tabular}{p{6.7cm} p{7.8cm}}
   \\[-0.4cm] \hline \hline \\[-0.4cm]
   Type  & Short characterization
   \\[0.1cm]  \hline  \\[-0.25cm]
   Paarerzeugung$^1$ & ...  \\[0.1cm]
   ...               & ...  \\[0.1cm]
   \hline \hline
\end{tabular}
\end{center}
\vspace{0.2cm}
\end{small}
\end{table}
\footnotetext[1]{A complete list of all ... }


%
%
%
%
%
\section{Beispiele/Examples}  


%
%
%
%
%
\section{...}  


%
%
%
%
%
\section{Zusammenfassung/Summary and outlook}  \label{sec_outlook}

A short overview is given ...


%
%
%
%
%
\begin{thebibliography}{99}

\bibitem{Andrews/Babiker:2012}
   H.~F.~Beyer und V.~P.~Shevelko (Eds.), \textit{Atomic Physics with Heavy Ion} (Springer, Berlin, Heidelberg u.a., 1999).

\bibitem{Grover:97}
   L.~K.~Grover, Phys.\ Rev.\ Lett.\ 79 (1997) 325.

\bibitem{Redfern:96}
   D.~Redfern, \textit{The Maple Handbook} (Springer, New York, 1996).

\end{thebibliography}








\newpage
%
%
%
%
%

\appendix

\section*{Anhang A / Appendix A: \hspace{0.02cm} Derivation of ...}



\end{document}
